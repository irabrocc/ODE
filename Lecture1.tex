% 声明为子文件,指定主文件
\documentclass[main.tex]{subfiles}

\begin{document}
\pagestyle{plain}
\setcounter{chapter}{0}

\chapter{First order equations. Different solution methods for different types of equations}
\label{chap:chapter1}
\subsection{Introduction}
A differential equation is an equation that involves a function and its derivatives, where the unknown is the function itself. 

\begin{example}
    \begin{enumerate}
        \item A falling object. The equation 
    \begin{equation}\label{eq:dropping}
        m\dfrac{dv}{dt} = mg - \gamma v
    \end{equation}
    describes the velocity $v$ of an object falling in the atmosphere near the sea level. Here, $m$ represents the mass of the object, $g$ is the acceleration due to gravity, and $\gamma$ is known as the drag coefficient. Always remind that $v = v(t)$. Here, $v$ is unknown and $m,g,r$ is known data.  
    \item Mice and owls. The equation 
    \begin{equation}
        \dfrac{dp}{dt} = rp - k
    \end{equation}
    describes the population of mice that inhabitate a rural area, in presence of owls. Here, $r$ represents the growth of the mice population and $k$ represents the rate of predation by owls. Here $p = p(t)$ is unknown and $r,k$ is known data. 
    \item Heat equation. The equation 
    \begin{equation}
        \dfrac{dT}{dt} - d \Delta T = 0
    \end{equation}
    represents the conduction of heat in an isotropic and homogeneous medium. Here, $\Delta$ is called the Laplace operator, and it's given by 
    \begin{equation}
        \Delta T = \dfrac{\partial^2 T}{\partial x^2} + \dfrac{\partial^2 T}{\partial y^2} + \dfrac{\partial^2 T}{\partial z^2}
    \end{equation}
    \end{enumerate}
\end{example}
\par Often, in practice, mathematical models given by differential equations are supplemented by associated conditions. For example, if we drop and object from certain height, then the phenomena is governed by the equation \eqref{eq:dropping} and the initial condition $v(0) = 0$. A very simple model for heat conduction in a long thin bar is given by 
\begin{equation}
    T'' = 0, \ \ \ \ 0 < x < L
\end{equation}
If we know the temperature of the extremes of the bar, we supplement this equation with the conditions $T(0) = a$ and $T(L) = b$. Differential problems supplemented with initial conditions are called initial value problems and those supplemented with boundary conditions are called boundary value problems. 
\par \noindent \textbf{Example} A problem of heat conduction in a 3-dimensional ball $D \subseteq \mathbb{R}^3$ is: 
\begin{equation}
    \begin{cases}
        \dfrac{\partial T}{\partial t} - \Delta T = 0, \ \ \ \ &x \in D, t > 0 \\
        T(x,0) = f(x), \ \ \ \ &x \in D (initial\ condition)\\
        T(x,t) = g(x,t), \ \ \ \ &x \in \partial D, t > 0 (boundary\ condition)
    \end{cases}
\end{equation}
\subsection{Classification of differential equations}
\begin{enumerate}
    \item Ordinary and partial differential equations. If the unknown depends on one variable, we say that the equation is an ordinary differential equation (ODE). If, instead, the unknown depends on several variables, we say that the equation is a partial differential equation (PDE). 
    \item Scalar differential equations and systems of differential equations. If we have a problem with one equation in one unknown, we say that the equation is scalar. If we have a problem with several equations in several unknowns, we say that the equation is a system of differential equations. For example, the system$$ 
    \begin{cases}
        \dfrac{dx}{dt} = a x - \alpha x y \\
        \dfrac{dy}{dt} = -cy + \delta x y
    \end{cases}
    $$
    where $x, y$ are unknowns and $a, \alpha, c, \delta$ are known data, is a system of differential equations which is known as the Lotka-Volterra system and describes the interaction between preys and predators. 
    \item The order of an ODE is the order of the highest derivative of the unknown. 
    \item Linear and non-linear equations. An ODE $F(t, u, u', u'', \ldots, u^{(n)}) = 0$ is said to be linear if $F$ is linear with respect to the last $(n - 1)$ variables. Otherwise, the equation is called non-linear. 
\end{enumerate}

\subsection{Well-posed problems}
Recall our first example \eqref{eq:dropping}. 
$$
\begin{cases}
    m \dfrac{dv}{dt} = mg - \gamma v, & t>0\\
    v(0) = v_0.
\end{cases}
$$

\par We say that our initial value problem for an ODE is well-posed if the following three conditions are satisfied:
\begin{enumerate}
    \item Existence: there exists a solution $v = v(t)$ that satisfies the equation and the initial condition. 
    \item Uniqueness: there is no other solution that satisfies the equation and the initial condition. 
    \item Stability: Small changes in the initial condition $v_0$ produce only small changes in the solution $v(t)$. 
\end{enumerate}

\begin{example}
    \begin{enumerate}
        \item Consider the initial value problem given by 
        $$\begin{cases}
            \dfrac{dy}{dt} = y & t > 0 \\
            y(0) = y_0
        \end{cases}$$
        If we integrate the equation, we find that $y(t) = Ce^t$. Also from the initial condition we have that $y(0) = C = y_0$. Thus, the solution of the initial value problem is $y(t) = y_0 e^t$. It can be also shown that this solution is unique in a course later. 
        \par Assume, for example, that $y_0 = 0$. So, the solution is the trivial solution $y(t) = 0$. Additionally, consider a "small" change in the initial data. 
        $$
        \begin{cases}
            \dfrac{dy}{dt} = y & t > 0 \\
            y(0) = \epsilon
        \end{cases}
        $$ 
        for some $0< \epsilon << 1$. Now the solution is $y(t) = \epsilon e^t$. In order to the problem has the stability condition we must have that $y_{\epsilon}$ is closed to $y$ for all $t > 0$. In other words, that $|y(t) - y_{\epsilon}(t)| \delta \forall t > 0$ for some small $\delta > 0$. But this is not true since 
        $$
        |y(t) - y_{\epsilon}(t)| = |\epsilon e^t| = \epsilon e^t \to \infty \ as \ t \to \infty.
        $$
        Thus, the problem does not satisfy the stability condition, so it's ill-posed(not well-posed). 
        \item Consider the initial value problem 
        $$
        \begin{cases}
            \dfrac{dy}{dt} = \sqrt{|y|} & t > 0 \\
            y(0) = 0
        \end{cases}
        $$ 
        This problem admits the trivial solution $y(t) = 0$. But it also admits the solution $y(t) = \dfrac{t^2}{4}$. Moreover, any function of the form 
        $$
        g(t) = \begin{cases}
            0 & t \in [0, c] \\
            \dfrac{(t-c)^2}{4} & t \in [c, \infty)
        \end{cases}
        $$
        is also a solution, for any $c \geq 0$. Since the problem has infinitely many solutions, it is ill-posed. 
    \end{enumerate}
\end{example}

\subsection{Some solution methods for first-order equations}
\par A first-order ODE is an equation of the form: 
\begin{equation}
    \dfrac{dy}{dt} = f(t,y)
\end{equation}
\par A linear first order ODE is one of the form: 
\begin{equation}\label{eq:linearfirstODE}
    \dfrac{dy}{dt} + a(t)y = b(t)
\end{equation}
Assume that $a(t)$ and $b(t)$ are continuous functions in $\mathbb{R}$. If we can find an equivalent equation of equation \ref{eq:linearfirstODE}, that is 
\begin{equation}\label{eq:simpleODE}
    \dfrac{dy}{dt} + a(t)y = b(t) \leadsto \dfrac{d \tilde{y}}{dt} = \tilde{b}(t)
\end{equation}
we are done, because we can solve \ref{eq:simpleODE} by integration. The idea is to look first for a function $\mu$ such that 
\begin{equation}\label{eq:qua}
    \mu (t) (\dfrac{dy}{dt} + a(t)y) = \mu(t) b(t)
\end{equation}
can be easily written in the form \ref{eq:simpleODE}. 
 The core of the integrating factor method is the following observation 
\begin{equation}
    \dfrac{d}{dt}(\mu(t) y(t)) = \mu(t) \dfrac{dy}{dt} + \mu'(t) y(t)
\end{equation} 
 Then if $\mu$ satisfies 
\begin{equation}
    \dfrac{d\mu }{dt} = a(t) \mu 
\end{equation}
then equation \ref{eq:qua} reduces to 
\begin{equation}
    \dfrac{d(\mu(t)y)}{dt} = \mu(t) b(t)
\end{equation}
Additionally, we can find $\mu$ by solving $\dfrac{d\mu}{dt} = a(t) \mu$. 
 Assume that this equation admits a solution $\mu$ which satisfies $\mu(t) > 0$ for all $t$. Then 
\begin{equation}
    \dfrac{1}{\mu} \dfrac{d\mu}{dt}  = a(t)
\end{equation}
Integrating both sides, we get 
\begin{equation}
    \int \dfrac{d}{dt} \ln(|\mu(t)|) dt = \int a(t) dt
\end{equation}
which implies 
\begin{equation}
    \ln(|\mu(t)|) = \int a(t) dt + C
\end{equation}
or equivalently 
\begin{equation}
    |\mu(t)| = e^C e^{\int a(t) dt}
\end{equation}
 Since we assume that $\mu(t) > 0$ for all $t$, we get that 
\begin{equation}
    \mu(t) = C e^{\int a(t) dt}
\end{equation}
for some $C > 0$. We can choose $C = 1$. So we have 
\begin{equation}
    \mu(t) = e^{\int a(t) dt}
\end{equation}
 It's easy to veryfy that $\mu$ is a solution to $\dfrac{du}{dt} = a(t) u$. 
 Then, equation \ref{eq:qua} can be written as 
\begin{equation}
    \dfrac{d}{dt} (e^{\int a(t) dt} y) = e^{\int a(t) dt} b(t)
\end{equation}
Integrating with respect to $t$ yields 
\begin{equation}
    e^{\int a(t) dt} y = \int e^{\int a(t) dt} b(t) dt + C
\end{equation}
Therefore, 
\begin{equation}\label{eq:finalForm}
    y(t) = e^{-\int a(t) dt} (\int e^{\int a(t) dt} b(t) dt + C)
\end{equation}
In summary, we deduced that if \ref{eq:linearfirstODE} has a solution, then the solution must be a function in the form \ref{eq:finalForm}. So, formula (10) is called the general solution to the linear first-order ODE \ref{eq:linearfirstODE}. Here $\mu$ is called an integrating factor. 
\end{document}